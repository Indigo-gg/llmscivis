% VTKSearcherV2 检索算法 - LaTeX Algorithm2e 格式
% 使用 \usepackage[ruled,vlined]{algorithm2e} 在文档中引入此代码
% 或使用 \usepackage[lined]{algorithm2e} 版本

\begin{algorithm}[ht]
\SetKwInOut{Input}{Input}
\SetKwInOut{Output}{Output}
\SetKwInOut{Require}{Require}
\SetKwFunction{AnalyzeQuery}{AnalyzeQuery}
\SetKwFunction{FindDocsByModules}{FindDocsByModules}
\SetKwFunction{BuildPrompt}{BuildPrompt}
\SetKwFunction{CalculateScores}{CalculateScores}
\SetKwFunction{GetRankedResults}{GetRankedResults}
\label{alg:vtksercher_v2}
\caption{VTKSearcherV2:加权关键词检索与重排序}

\Input{
  $query$: 原始用户完整请求字符串 \\
  $queryList$: 分割后的子查询列表,每项包含 $description$ 和 $weight$
}
\Output{
  $finalPrompt$: 包含Top-6示例的最终提示词字符串
}
\Require{
  $MongoDB$ 连接已初始化且包含代码库 \\
  每个文档包含 $meta\_info$ 字段和 $vtkjs\_modules$ 模块列表
}

\tcp{====== 阶段1: 广度召回 (Recall) ======}
$candidateDocs \gets \{\}$ \tcp*[r]{用于去重的字典}
$rawHistory \gets []$ \tcp*[r]{记录每个子查询的原始召回结果}

\ForEach{$queryItem$ \textbf{in} $queryList$}{
  \tcp{解析子查询文本}
  $qText \gets queryItem.description$ \\
  $qWeight \gets queryItem.weight$ \\
  
  \tcp{分析查询提取关键词模块}
  $analyzedData \gets$ \AnalyzeQuery{$qText$} \\
  $qModules \gets analyzedData.modules$ \label{alg:analyze_query} \\
  
  \tcp{数据库检索相关文档}
  $docs \gets$ \FindDocsByModules{$qModules$} \label{alg:db_query} \\
  
  \tcp{记录原始召回结果}
  $docsForHistory \gets []$ \\
  \ForEach{$doc$ \textbf{in} $docs$}{
    $docCopy \gets doc.copy()$ \\
    $docCopy.queryDescription \gets qText$ \\
    $docsForHistory.append(docCopy)$ \\
    
    \tcp{添加到去重候选池}
    $docId \gets doc.faissId \textbf{ or } doc.filePath$ \\
    \If{$docId \notin candidateDocs$}{
      $candidateDocs[docId] \gets doc$ \\
    }
  }
  $rawHistory.\text{append}(docsForHistory)$ \\
}

$candidateList \gets candidateDocs.values()$ \\
print "Total unique candidates recalled: " $|candidateList|$ \\

\tcp{====== 阶段2: 深度精排 (Rerank) ======}
\tcp{第一步: 权重归一化}
$totalWeight \gets 0$ \\
$validQueries \gets []$ \\

\ForEach{$q$ \textbf{in} $queryList$}{
  $w \gets q.\text{weight}$ \textbf{ or } $5.0$ \\
  $totalWeight \gets totalWeight + w$ \\
  $q.\text{parsedWeight} \gets w$ \\
  $validQueries.\text{append}(q)$ \\
}

\If{$totalWeight = 0$}{
  $totalWeight \gets 1$ \tcp*[r]{防止除零}
}

\tcp{第二步: 遍历子查询计算文档得分}
$docScores \gets \{\}$ \tcp*[r]{文档得分字典}
$docDetails \gets \{\}$ \tcp*[r]{文档匹配详情}

\ForEach{$queryItem$ \textbf{in} $validQueries$}{
  $qText \gets queryItem.description$ \\
  $qWeight \gets queryItem.\text{parsedWeight}$ \\
  
  \tcp{计算该子查询的归一化权重}
  $normalizedWeight \gets \frac{qWeight}{totalWeight}$ \label{alg:weight_norm} \\
  
  \tcp{提取查询关键词}
  $analyzedData \gets$ \AnalyzeQuery{$qText$} \\
  $qModules \gets analyzedData.\text{modules}$ \\
  
  \If{$qModules = \emptyset$}{
    \textbf{continue} \\
  }
  
  \tcp{遍历每个候选文档进行打分}
  \ForEach{$doc$ \textbf{in} $candidateList$}{
    $docId \gets doc.faissId \textbf{ or } doc.filePath$ \\
    
    \tcp{计算文档对该查询的关键词命中数}
    $(hits, matchedKeywords) \gets$ \texttt{CountHits}$(doc, qModules)$ \label{alg:count_hits} \\
    
    \If{$hits > 0$}{
      \tcp{累加得分: 得分增量 = 命中数 × 归一化权重}
      $scoreIncrement \gets hits \times normalizedWeight$ \label{alg:scoring_formula} \\
      $docScores[docId] \gets docScores[docId] + scoreIncrement$ \\
      
      \tcp{记录匹配详情}
      \If{$docId \notin docDetails$}{
        $docDetails[docId] \gets \{\text{docObj}: doc, \text{matches}: [], \text{keywords}: \emptyset\}$ \\
      }
      $docDetails[docId].\text{matches}.\text{append}(\text{"Query: "} + qText)$ \\
      $docDetails[docId].\text{keywords} \gets docDetails[docId].\text{keywords} \cup matchedKeywords$ \\
    }
  }
}

\tcp{第三步: 排序输出Top-K结果}
$rankedList \gets []$ \\

\ForEach{$(docId, score)$ \textbf{in} $docScores.items()$}{
  $doc \gets docDetails[docId].\text{docObj}$ \\
  $doc.\text{rerankScore} \gets score$ \\
  $doc.\text{matchExplanation} \gets docDetails[docId].\text{matches}$ \\
  $doc.\text{matchedKeywords} \gets docDetails[docId].\text{keywords}$ \\
  $rankedList.\text{append}(doc)$ \\
}

\tcp{按得分降序排序}
$rankedList.\text{sort}(\text{key}=\lambda x: x.\text{rerankScore}, \text{reverse}=\text{True})$ \label{alg:sort_by_score} \\

$finalResults \gets rankedList[0:6]$ \label{alg:top_k} \\

\Return{$finalResults$}

\end{algorithm}


% ============================================================
% 算法说明文档
% ============================================================

\section*{算法说明与设计细节}

\subsection*{总体架构}

VTKSearcherV2采用\textbf{两阶段检索架构}:
\begin{itemize}
  \item \textbf{第一阶段 - 广度召回 (Recall)}:对每个子查询独立执行数据库查询,广泛召回所有可能相关的文档,然后进行去重
  \item \textbf{第二阶段 - 深度精排 (Rerank)}:基于子查询权重,对所有候选文档进行加权打分,精准排序输出Top-6结果
\end{itemize}

\subsection*{权重归一化公式(第二阶段 - 第一步)}

\textbf{目标}:确保每个子查询的权重被转换为相对重要性系数(0.0 ~ 1.0)

\textbf{公式}:
\begin{equation}
\text{normalizedWeight}_i = \frac{w_i}{\sum_{j=1}^{n} w_j}
\label{eq:weight_normalization}
\end{equation}

其中:
\begin{itemize}
  \item $w_i$ 为第 $i$ 个子查询的原始权重
  \item $\sum_{j=1}^{n} w_j$ 为所有 $n$ 个子查询的权重总和
  \item 若总和为0,默认设置为1(防止除零错误)
\end{itemize}

\subsection*{文档得分计算公式(第二阶段 - 第二步)}

\textbf{目标}:根据文档对所有子查询的综合匹配程度进行打分

\textbf{公式}:
\begin{equation}
\text{Score}(\text{doc}) = \sum_{i=1}^{n} (\text{hits}_i \times \text{normalizedWeight}_i)
\label{eq:document_score}
\end{equation}

其中:
\begin{itemize}
  \item $\text{hits}_i$ 为文档与第 $i$ 个子查询的关键词命中数(整数值 $\geq 0$)
  \item $\text{normalizedWeight}_i$ 为第 $i$ 个子查询的归一化权重
  \item 最终得分为各子查询贡献的累加和
\end{itemize}

\textbf{设计理由}:
\begin{itemize}
  \item \textbf{累加机制}:文档能解决的子查询越多,总分越高,体现"综合相关性"
  \item \textbf{权重控制}:重要性高的子查询权重大,其匹配贡献度大
  \item \textbf{线性组合}:简单高效,避免过度复杂的非线性计算
\end{itemize}

\subsection*{数值示例}

\textbf{假设场景}:
\begin{itemize}
  \item 子查询1: "render a cone" (权重=10)
  \item 子查询2: "set background to blue" (权重=3)
  \item 总权重 = 10 + 3 = 13
\end{itemize}

\textbf{权重归一化}:
\begin{itemize}
  \item 子查询1归一化权重 = $\frac{10}{13} \approx 0.769$
  \item 子查询2归一化权重 = $\frac{3}{13} \approx 0.231$
\end{itemize}

\textbf{文档A的得分计算}:假设:
\begin{itemize}
  \item 匹配子查询1的关键词:2个 ($\text{hits}_1 = 2$)
  \item 匹配子查询2的关键词:1个 ($\text{hits}_2 = 1$)
\end{itemize}

\textbf{最终得分}:
\begin{equation}
\text{Score}(docA) = 2 \times 0.769 + 1 \times 0.231 = 1.538 + 0.231 = 1.769
\end{equation}

\subsection*{关键词匹配策略(CountHits算法)}

关键词匹配采用\textbf{双策略}设计:

\textbf{策略A - 模块列表匹配}(优先级高):
\begin{itemize}
  \item \textbf{精确匹配}:$qmLower = dm$(忽略大小写)
  \item \textbf{后缀匹配}:$dm.\text{endsWith}(qmClean)$(解决命名空间问题)
  \item 例如:查询词 "vtkActor" 可匹配 "vtk.Rendering.Core.vtkActor"
\end{itemize}

\textbf{策略B - 描述文本匹配}(降级补充):
\begin{itemize}
  \item 如果模块列表未命中,检查文档描述文本中是否包含该关键词
  \item 提高了非结构化匹配的召回率
\end{itemize}

\subsection*{数据库查询设计}

\textbf{查询方式}:使用MongoDB的正则表达式后缀匹配
\begin{itemize}
  \item 模式:$\text{regex}(\text{module}\$$, \text{re.IGNORECASE})$
  \item 作用:匹配字符串结尾是该模块名的情况(忽略大小写)
  \item 优势:一次性批量查询所有包含任意一个查询模块的文档
\end{itemize}

\subsection*{去重与合并策略}

在广度召回阶段:
\begin{itemize}
  \item 使用字典 $candidateDocs$ 按 $docId$(优先FAISS ID,其次文件路径)进行去重
  \item 同一文档即使被多个子查询召回,也只保留一份
  \item 避免重复处理相同文档的打分逻辑
\end{itemize}

\subsection*{算法复杂度分析}

\textbf{输入规模}:
\begin{itemize}
  \item $n$:子查询数量
  \item $m$:候选文档数量(去重后)
  \item $k$:每个文档的模块数量(平均)
  \item $d$:每个查询提取的关键词数量(平均)
\end{itemize}

\textbf{复杂度}:
\begin{itemize}
  \item 广度召回阶段:$O(n \times \text{DB\_query\_time})$
  \item 权重归一化:$O(n)$
  \item 深度精排阶段:$O(n \times m \times d \times k)$ 
    \begin{itemize}
      \item $n$ 个子查询遍历
      \item $m$ 个候选文档
      \item 每次CountHits操作复杂度为 $O(d \times k)$
    \end{itemize}
  \item 排序阶段:$O(m \log m)$
  \item \textbf{总体}:$O(n \times m \times d \times k + m \log m)$,实际通常 $d, k$ 较小(10-20以内)
\end{itemize}

\subsection*{关键设计优势}

\begin{enumerate}
  \item \textbf{权重感知}:子查询权重直接控制其对最终排序的影响力
  \item \textbf{综合评估}:文档被评估是否"整体上解决问题",而非单一维度评分
  \item \textbf{透明可解释}:每个文档的得分来源清晰记录(match\_explanation)
  \item \textbf{灵活扩展}:可轻松调整权重或添加新的匹配策略
  \item \textbf{高效实现}:避免向量计算,纯关键词匹配,响应快速
\end{enumerate}

\subsection*{与Embedding检索的对比}

本算法采用\textbf{加权关键词检索}替代Embedding相似度检索的原因:
\begin{itemize}
  \item \textbf{速度快}:无需计算和存储Embedding向量
  \item \textbf{权重控制精准}:显式权重调整比向量空间距离更直观
  \item \textbf{可解释性强}:每次匹配都有明确的规则和记录
  \item \textbf{VTK.js特化}:关键词匹配天然适合模块名这类离散符号
  \item \textbf{成本低}:无需预计算Embedding,节省存储和计算资源
\end{itemize}
